\documentclass{article}
\usepackage{geometry}
\usepackage[utf8]{inputenc}

\usepackage{fancyhdr}
\pagestyle{fancy}
\hoffset=0.3in
\voffset=-0.2in
\headsep=0.5in
\lhead{}
\chead{}
\rhead{\thepage}
\lfoot{}
\cfoot{}
\rfoot{}
\renewcommand{\headrulewidth}{0pt}
\renewcommand{\footrulewidth}{0pt}
\setlength{\parindent}{4em}
\paperheight=11in
\textheight=9.0in
\usepackage[pdftex]{graphicx}
\large

\usepackage{titlesec}
\titleformat*{\section}{\LARGE}
\titleformat*{\subsection}{\Large}
\titleformat*{\subsubsection}{\Large}

\usepackage{alltt}

\linespread{2}
\usepackage{setspace}
\usepackage{cite}


% declare the path(s) where your graphic files are
\graphicspath{{./images/}}
 % and their extensions so you won't have to specify these with
 % every instance of \includegraphics
 \DeclareGraphicsExtensions{.pdf,.jpeg,.png}

\usepackage{threeparttable}

\usepackage{enumerate}

\renewcommand*\contentsname{\empty}
\renewcommand{\listtablename}{\empty}
\renewcommand{\listfigurename}{\empty}
\renewcommand\refname{}



\begin{document}

\pagenumbering{arabic}
\textbf{\section{Programs and Expressions}}
\textbf{\section{Structures}}
\textbf{\section{String Processing}}
\textbf{\subsection{String Indexes}}
\textbf{\subsection{Character Sets}}
\textbf{\subsection{Character Escaptes}}
\textbf{\subsection{String Scanning}}
\textbf{\subsection{Pattern Matching}}
Unicon has a SNOBOL4 type pattern matching facility.  Combined with the string scanning facilities they are highly flexible and readable string processing facilities.  In this section you will learn

\begin{itemize}
\item how to initialize a pattern match
\item how to construct a pattern
\item Unicon's pattern operators
\item Unicon's primitive pattern functions
\end{itemize}

\textbf{\subsubsection*{pattern match operation}}
A pattern match operator operands consist of a \textit{subject} string and a pattern which was defined earlier.  It is expressed in the form:\\

\begin{verbatim}
     subject ?? pattern
\end{verbatim}

The \texttt{??} operator initializes the pattern match, searches the \textit{subject} string on the left for the first occurrence of the pattern on the right.  This is an unanchord pattern match.  If found, it suspends the substring matching the pattern.  Should there be more than one occurrence of the pattern, it functions as a generator suspending each subsequent matching substring as requested.

When a pattern match is part of an expression in the string scanning environment the unary \texttt{=} operator is used.  In the following example the string scanning environment is initialized with the \texttt{?} operator.  The \texttt{\&pos} is set to 1 or the beginning of the \textit{subject} string.  An anchored pattern match is performed on the subject string from the \texttt{\&pos}.  In anchored pattern matches, the match must begin at the current \texttt{\&pos} otherwise it fails.  If it succeeds, then \texttt{result} will be assigned the matching substring.\\

\begin{verbatim}
     subject ? {
        result := =pattern
     }
\end{verbatim}

\textbf{\subsubsection*{pattern construction}}
A pattern can be as simple as a single character or a string, and can be a very complex make up of pattern operators, pattern functions, procedure and method calls, and other patterns.


\end{document}
